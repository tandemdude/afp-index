\documentclass[11pt,a4paper]{article}
\usepackage[utf8]{inputenc}
\usepackage{isabelle,isabellesym}
\usepackage{amsmath}
\usepackage{amssymb}
\usepackage[english]{babel}

% this should be the last package used
\usepackage{pdfsetup}

% urls in roman style, theory text in math-similar italics
\urlstyle{rm}
\isabellestyle{it}

\newcommand{\etal}{\emph{et al.}}
\newcommand{\keyword}[1]{\ensuremath{\textsf{\textbf{#1}}}}
\newcommand{\Klemma}{\keyword{lemma}}
\newcommand{\Ktheorem}{\keyword{theorem}}
\newcommand{\Kif}{\keyword{if}}
\newcommand{\Klet}{\keyword{let}}
\newcommand{\Kleft}{\boldsymbol \leftarrow}
\newcommand{\Kdo}{\keyword{do}}
\newcommand{\Kin}{\keyword{in}}
\newcommand{\Kthen}{\keyword{then}}
\newcommand{\Kelse}{\keyword{else}}

\newcommand{\identifier}[1]{\textsl{\textsf{#1}}}
\newcommand{\Iaexp}{\identifier{aexp}}
\newcommand{\IAffine}{\identifier{Affine}}
\newcommand{\IAffines}{\identifier{Affines}}
\newcommand{\Iaffineofivl}{\identifier{affine-of-ivl}}
\newcommand{\IAdd}{\identifier{Add}}
\newcommand{\Iaddaffine}{\identifier{add-affine}}
\newcommand{\IAddE}{\identifier{AddE}}
\newcommand{\IBasis}{\identifier{Basis}}
\newcommand{\Ibinop}{\identifier{binop}}
\newcommand{\Ibox}{\identifier{box}}
\newcommand{\Icenter}{\identifier{center}}
\newcommand{\Icoeff}{\identifier{coeff}}
\newcommand{\Icoeffs}{\identifier{coeffs}}
\newcommand{\Iapprox}{\identifier{approx}}
\newcommand{\ID}{\identifier{D}}
\newcommand{\Ideg}{\identifier{deg}}
\newcommand{\Idim}{\identifier{dim}}
\newcommand{\Idivl}{\identifier{div}^-}
\newcommand{\Idivr}{\identifier{div}^+}
\newcommand{\Ieexp}{\identifier{eexp}}
\newcommand{\IElem}{\identifier{elem}}
\newcommand{\Ieulerseries}{\identifier{euler-series}}
\newcommand{\Ieulerstep}{\identifier{euler-step}}
\newcommand{\Ifalse}{\identifier{False}}
\newcommand{\Ifilter}{\identifier{filter}}
\newcommand{\Ifold}{\identifier{fold}}
\newcommand{\Ifor}{\identifier{for}}
\newcommand{\Ifst}{\identifier{fst}}
\newcommand{\IFloat}{\identifier{Float}}
\newcommand{\Iindices}{\identifier{indices}}
\newcommand{\IInverse}{\identifier{Inverse}}
\newcommand{\Iinverseaffine}{\identifier{inverse-affine}}
\newcommand{\Iivp}{\identifier{ivp}}
\newcommand{\Ilen}{\identifier{len}}
\newcommand{\Imap}{\identifier{map}}
\newcommand{\Imerge}{\identifier{merge}}
\newcommand{\IMinus}{\identifier{Minus}}
\newcommand{\Iminusaffine}{\identifier{minus-affine}}
\newcommand{\IMult}{\identifier{Mult}}
\newcommand{\Imultaffine}{\identifier{mult-affine}}
\newcommand{\INone}{\identifier{None}}
\newcommand{\INum}{\identifier{Num}}
\newcommand{\Irad}{\identifier{rad}}
\newcommand{\Iradup}{\identifier{rad}^+}
\newcommand{\Iround}{\identifier{round}}
\newcommand{\IScale}{\identifier{Scale}}
\newcommand{\Iscaleaffine}{\identifier{scale-affine}}
\newcommand{\Isol}{\identifier{sol}}
\newcommand{\ISome}{\identifier{Some}}
\newcommand{\Isplit}{\identifier{split}}
\newcommand{\Isummarize}{\identifier{summarize}}
\newcommand{\To}{\Rightarrow}
\newcommand{\Itrue}{\identifier{True}}
\newcommand{\Itruncatedown}{\identifier{trunc}^-}
\newcommand{\Itruncateup}{\identifier{trunc}^+}
\newcommand{\Iroundbinop}{\identifier{round-binop}}
\newcommand{\UNIV}[1]{\mathcal{U}_{#1}}
\newcommand{\Iunzip}{\identifier{unzip}}
\newcommand{\IVar}{\identifier{Var}}
\newcommand{\Izip}{\identifier{zip}}

\newcommand{\Tset}[1]{#1\,\identifier{set}}
\newcommand{\Tlist}[1]{#1\,\identifier{list}}
\newcommand{\Taffine}[1]{#1\,\identifier{affine}}
\newcommand{\Toption}[1]{#1\,\identifier{option}}
\newcommand{\Tbcontfun}[2]{#1\To_{\identifier{bc}}#2}
\newcommand{\Tfinmap}[2]{#1\rightharpoondown_{\identifier{f}}#2}
\newcommand{\Tfilter}[1]{#1\,\identifier{filter}}

\newcommand{\Bool}{\ensuremath{\mathbb{B}}}
\newcommand{\Real}{\ensuremath{\mathbb{R}}}
\newcommand{\Float}{\ensuremath{\mathbb{F}}}
\newcommand{\Eucl}[1]{\ensuremath{\mathbb{R}^{#1}}}
\newcommand{\Complex}{\ensuremath{\mathbb{C}}}
\newcommand{\Nat}{\ensuremath{\mathbb{N}}}
\newcommand{\Integer}{\ensuremath{\mathbb{Z}}}

\newcommand{\ToDO}[1]{{\color{red}\textbf{TODO:} #1}}
\newcommand{\Todo}[1]{{\color{red}[#1]}}

\newcommand{\limseq}{\xrightarrow{\hspace*{2em}}}
\newcommand{\interpret}[1]{\ensuremath{[\![#1]\!]}}

\title{Formally Verified Computation of Enclosures of Solutions of Ordinary
Differential Equations}
\author{Fabian Immler}

\begin{document}

\maketitle

\begin{abstract}

Ordinary differential equations (ODEs) are ubiquitous when modeling continuous
dynamics. Classical numerical methods compute approximations of the solution,
however without any guarantees on the quality of the approximation.
Nevertheless, methods have been developed that are supposed to compute
enclosures of the solution.

In this paper, we demonstrate that enclosures of the solution can be verified
with a high level of rigor: We implement a functional algorithm that computes
enclosures of solutions of ODEs in the interactive theorem prover Isabelle/HOL,
where we formally verify (and have mechanically checked) the safety of the
enclosures against the existing theory of ODEs in Isabelle/HOL.

Our algorithm works with dyadic rational numbers with statically fixed precision
and is based on the well-known Euler method. We abstract discretization and
round-off errors in the domain of affine forms. Code can be extracted from the
verified algorithm and experiments indicate that the extracted code exhibits
reasonable efficiency.

\end{abstract}

\section{Relations to the paper}

Here we relate the contents of our NFM~2014 paper~\cite{Immler2014} with the sources you find here.
In the following list we show which notions and theorems in the
paper correspond to which parts of the source code. If you are (still) interested in the relations
to our ITP~2012 paper~\cite{immlerhoelzl}, you should take a look at the document of older releases
 (before Isabelle~2013-1) of this AFP entry.

\begin{enumerate}
\item Introduction
\item Background
  \begin{enumerate}
  \item Real numbers: Representation of real numbers with dyadic floats is set up in the separate
    entry Affine Arithmetic~\cite{Immler2014b}
  \item Euclidean Space: definition in image Multivariate-Analysis
  \item Derivatives: definition in Multivariate-Analysis
  \item Notes on Taylor Series Expansion in Euclidean Space: A formal proof of a similar problem
    with just the mean value theorem is given in Section~\ref{sec:countermvt}
  \item Ordinary Differential Equations
    \begin{itemize}
    \item Definition~1: Definition \textit{ivp} in Section~\ref{sec:solutions}
    \item Definition~2: Definition \textit{solution} in Section~\ref{sec:solutions}
    \item Theorem~3: In Section~\ref{sec:plclosed} resp. Section~\ref{sec:ivp-ubs}
    \item Theorem~4: In Section~\ref{sec:setconsistent}
    \end{itemize}
  \end{enumerate}
\item Affine Arithmetic: see the separate entry Affine Arithmetic~\cite{Immler2014b}
\item Approximation of ODEs:\\
  Assumptions are in locales \textit{approximate-ivp} and \textit{approximate-sets} in
  Section~\ref{sec:euleraform}
  \begin{enumerate}
  \item Euler Step: Definitions in locale \textit{approximate-ivp0} in Section~\ref{sec:euleraformcode} \\
    Theorem~7 and Theorem~8 are in Lemma \textit{unique-on-euler-step}
  \item Euler Series: Definitions in locale \textit{approximate-ivp0} in Section~\ref{sec:euleraformcode} \\
    Theorem~9 is Lemma \textit{intervals-of-accum}
  \end{enumerate}
\item Experiments: Oil reservoir problem in Section~\ref{sec:exampleoil}, Second example in
  Section~\ref{sec:example1}
\item Conclusion
\end{enumerate}

\tableofcontents

% sane default for proof documents
\parindent 0pt\parskip 0.5ex

% generated text of all theories
\input{session}

% optional bibliography
\bibliographystyle{abbrv}
\bibliography{root}

\end{document}

%%% Local Variables:
%%% mode: latex
%%% TeX-master: t
%%% End:
