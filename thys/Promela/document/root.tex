\documentclass[11pt,a4paper]{article}
\usepackage{isabelle,isabellesym}
\usepackage[english]{babel}
\usepackage{amssymb}
\usepackage[only,bigsqcap]{stmaryrd}

% this should be the last package used
\usepackage{pdfsetup}

% urls in roman style, theory text in math-similar italics
\urlstyle{rm}
\isabellestyle{it}

\newcommand{\isaheader}[1]{#1}
\newcommand{\isachapter}[1]{\chapter{#1}}
\newcommand{\isasection}[1]{\section{#1}}

% General
\newcommand{\ie}{i.\,e.\ }
\newcommand{\eg}{e.\,g.\ }
\newcommand{\wrt}{w.\,r.\,t.\ }
\newcommand{\cf}{cf.\ }

\begin{document}

\title{Promela Formalization}
\author{By Ren\'{e} Neumann}
\maketitle

\begin{abstract}
    We present an executable formalization of the language Promela, the description language for models of the model checker SPIN. This formalization is part of the work for a completely verified
    model checker (CAVA), but also serves as a useful (and executable!) description of the semantics of the language itself, something that is currently missing.
    The formalization uses three steps: It takes an abstract syntax tree generated from an SML parser, removes syntactic sugar and enriches it with type information. This further gets translated into a transition system, on which the semantic engine (read: successor function) operates.
\end{abstract}

\clearpage

\tableofcontents

\clearpage


\section{Introduction}

 
We formalize HyperCTL$^*$, a temporal logic for expressing security properties 
introduced in [1,2].  
We first define a shallow embedding of HyperCTL$^*$, within which we prove 
inductive and coinductive rules for the operators.  
Then we show that a HyperCTL$^*$ formula captures Goguen-Meseguer noninterference, a landmark 
information flow property.  
We also define a deep embedding and connect it to the shallow embedding 
by a denotational semantics, for which we prove sanity w.r.t.\ 
dependence on the free variables.  Finally, we show that under some finiteness assumptions 
about the model, 
noninterference is given by a (finitary) syntactic formula.

\par \ \par
For the semantics of HyperCTL$^*$, we mainly follow the earlier paper [1].  
The Kripke structure for representing noninterference is essentially that of [1,Appendix B] -- 
however, instead of using the formula from [1,Appendix B], we further add 
idle transitions to the Kripke structure and use the simpler formula from [1,Section 2.4].  


\par \ \par
[1] Bernd Finkbeiner, Markus N. Rabe and C\'{e}sar S\'{a}nchez.  
A Temporal Logic for Hyperproperties. CoRR, abs$/$1306.6657, 2013.

\par \ \par
[2] Michael R. Clarkson, Bernd Finkbeiner, Masoud Koleini, Kristopher K. Micinski, 
Markus N. Rabe and C\'{e}sar S\'{a}nchez. 
Temporal Logics for Hyperproperties. POST 2014, 265-284.  




% sane default for proof documents
\parindent 0pt\parskip 0.5ex

% generated text of all theories
\input{session}

% optional bibliography
\bibliographystyle{abbrv}
\bibliography{root}

\end{document}

%%% Local Variables:
%%% mode: latex
%%% TeX-master: t
%%% End:
