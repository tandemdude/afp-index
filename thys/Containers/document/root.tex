\documentclass[11pt,a4paper]{book}
\usepackage{isabelle,isabellesym}
\usepackage{amssymb}
\usepackage[english]{babel}
\usepackage[only,bigsqcap]{stmaryrd}
\usepackage{booktabs}

% this should be the last package used
\usepackage{pdfsetup}

% urls in roman style, theory text in math-similar italics
\urlstyle{rm}
\isabellestyle{it}

% Tweaks
\newcounter{TTStweak_tag}
\setcounter{TTStweak_tag}{0}
\newcommand{\setTTS}{\setcounter{TTStweak_tag}{1}}
\newcommand{\resetTTS}{\setcounter{TTStweak_tag}{0}}
\newcommand{\insertTTS}{\ifnum\value{TTStweak_tag}=1 \ \ \ \fi}

\renewcommand{\isakeyword}[1]{\resetTTS\emph{\bf\def\isachardot{.}\def\isacharunderscore{\isacharunderscorekeyword}\def\isacharbraceleft{\{}\def\isacharbraceright{\}}#1}}
\renewcommand{\isachardoublequoteopen}{\insertTTS}
\renewcommand{\isachardoublequoteclose}{\setTTS}
\renewcommand{\isanewline}{\mbox{}\par\mbox{}\resetTTS}

\renewcommand{\isamarkupcmt}[1]{\hangindent5ex{\isastylecmt --- #1}}

\newcommand{\isaheader}[1]{\section{#1}}

\makeatletter
\newenvironment{abstract}{%
  \small
  \begin{center}%
    {\bfseries \abstractname\vspace{-.5em}\vspace{\z@}}%
  \end{center}%
  \quotation}{\endquotation}
\makeatother

\begin{document}

\title{Light-Weight Containers}
\author{Andreas Lochbihler}
\maketitle

\begin{abstract}
  This development provides a framework for container types like sets and maps such that generated code implements these containers with different (efficient) data structures.
  Thanks to type classes and refinement during code generation, this light-weight approach can seamlessly replace Isabelle's default setup for code generation.
  Heuristics automatically pick one of the available data structures depending on the type of elements to be stored, but users can also choose on their own.
  The extensible design permits to add more implementations at any time.

  To support arbitrary nesting of sets, we define a linear order on sets based on a linear order of the elements and provide efficient implementations.
  It even allows to compare complements with non-complements.
\end{abstract}

\clearpage

\tableofcontents

\clearpage

% sane default for proof documents
\parindent 0pt\parskip 0.5ex

\chapter{Introduction}

This development focuses on generating efficient code for container types like sets and maps.
It falls into two parts: 
First, we define linear order on sets (Ch.~\ref{chapter:linear:order:set}) that is efficiently executable given a linear order on the elements.
Second, we define an extensible framework LC (for light-weight containers) that supports multiple (efficient) implementations of container types (Ch.~\ref{chapter:light-weight:containers}) in generated code.
Both parts heavily exploit type classes and the refinement features of the code generator \cite{HaftmannKrausKuncarNipkow2013ITP}.
This way, we are able to implement the HOL types for sets and maps directly, as the name light-weight containers (LC) emphasises.

In comparison with the Isabelle Collections Framework (ICF) \cite{LammichLochbihler2010ITP,Lammich2009AFP}, the style of refinement is the major difference.
In the ICF, the container types are replaced with the types of the data structures inside the logic.
Typically, the user has to define his operations that involve maps and sets a second time such that they work on the concrete data structures; then, she has to prove that both definitions agree.
With LC, the refinement happens inside the code generator.
Hence, the formalisation can stick with the types $'a\ set$ and $('a, 'b)\ mapping$ and there is no need to duplicate definitions or prove refinement.
The drawback is that with LC, we can only implement operations that can be fully specified on the abstract container type.
In particular, the internal representation of the implementations may not affect the result of the operations.
For example, it is not possible to pick non-deterministically an element from a set or fold a set with a non-commutative operation, i.e., the result depends on the order of visiting the elements.

For more documentation and introductory material refer to the userguide (Chapter~\ref{chapter:Userguide}) and the ITP-2013 paper \cite{Lochbihler2013ITP}.

% generated text of all theories
\input{session}

%\chapter{Conclusion}\label{ch:conclusion}

This work presented the Isabelle Collections Framework, an efficient and extensible collections framework for Isabelle/HOL.
The framework features data-refinement techniques to refine algorithms to use concrete collection datastructures,
and is compatible with the Isabelle/HOL code generator, such that efficient code can be generated for all supported target languages.
Finally, we defined a data refinement framework for the while-combinator, and used it to specify a state-space exploration algorithm
and stepwise refined the specification to an executable DFS-algorithm using a hashset to store the set of already known states.

Up to now, interfaces for sets and maps are specified and implemented using lists, red-black-trees, and hashing. Moreover, an amortized constant time 
fifo-queue (based on two stacks) has been implemented. However, the framwork is extensible, i.e. new interfaces, algorithms and implementations can easily be added and integrated with the existing ones.

\section {Trusted Code Base}
  In this section we shortly characterize on what our formal proofs depend, i.e. how to interpret the information contained in this formal proof and the fact that it
  is accepted by the Isabelle/HOL system.

  First of all, you have to trust the theorem prover and its axiomatization of HOL, the ML-platform, the operating system software and the hardware it runs on.
  All these components are, in theory, able to cause false theorems to be proven. However, the probability of a false theorem to get proven due to a hardware error 
  or an error in the operating system software is reasonably low. There are errors in hardware and operating systems, but they will usually cause the system to crash 
  or exhibit other unexpected behaviour, instead of causing Isabelle to quitely accept a false theorem and behave normal otherwise. The theorem prover itself is a bit more critical in this aspect. However, Isabelle/HOL is implemented in LCF-style, i.e. all the proofs are eventually checked by a small kernel of trusted code, containing rather simple operations. HOL is the logic that is most frequently used with Isabelle, and it is unlikely that it's axiomatization in Isabelle is inconsistent and no one has found and reported this inconsistency yet.

  The next crucial point is the code generator of Isabelle. We derive executable code from our specifications. The code generator contains another (thin) layer of untrusted code. This layer has some known deficiencies\footnote{For example, the Haskell code generator may generate variables starting with upper-case letters, while the Haskell-specification requires variables to start with lowercase letters. Moreover, the ML code generator does not know the ML value restriction, and may generate code that violates this restriction.} (as of Isabelle2009) in the sense that invalid code is generated. This code is then rejected by the target language's compiler or interpreter, but does not silently compute the wrong thing. 

  Moreover, assuming correctness of the code generator, the generated code is only guaranteed to be {\em partially} correct\footnote{A simple example is the always-diverging function ${\sf f_{div}}::{\sf bool} = {\sf while}~(\lambda x.~{\sf True})~{\sf id}~{\sf True}$ that is definable in HOL. The lemma $\forall x.~ x = {\sf if}~{\sf f_{div}}~{\sf then}~x~{\sf else}~x$ is provable in Isabelle and rewriting based on it could, theoretically, be inserted before the code generation process, resulting in code that always diverges}, i.e. there are no formal termination guarantees.

  Furthermore, manual adaptations of the code generator setup are also part of the trusted code base.
  For array-based hash maps, the Isabelle Collections Framework provides an ML implementation for arrays with in-place updates that is unverified; for Haskell, we use the DiffArray implementation from the Haskell library.
  Other than this, the Isabelle Collections Framework does not add any adaptations other than those available in the Isabelle/HOL library, in particular Efficient\_Nat.

\section{Acknowledgement}
We thank Tobias Nipkow for encouraging us to make the collections framework an independent development. Moreover, we thank Markus M\"uller-Olm for discussion about data-refinement. Finally, we thank the people on the Isabelle mailing list for quick and useful response to any Isabelle-related questions.


% optional bibliography
\bibliographystyle{abbrv}
\bibliography{root}

\end{document}

%%% Local Variables:
%%% mode: latex
%%% TeX-master: t
%%% End:
