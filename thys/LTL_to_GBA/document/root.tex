\documentclass[11pt,a4paper]{article}
\usepackage{isabelle,isabellesym}

\usepackage{amssymb}
\usepackage[english]{babel}

% this should be the last package used
\usepackage{pdfsetup}

% urls in roman style, theory text in math-similar italics
\urlstyle{rm}
\isabellestyle{it}


\begin{document}

\title{Converting Linear-Time Temporal Logic to Generalized B\"uchi Automata}
\author{Alexander Schimpf and Peter Lammich}
\maketitle

\begin{abstract}
We formalize linear-time temporal logic (LTL) and the algorithm by Gerth et al.\ to convert LTL formulas to generalized B\"uchi automata.
We also formalize some syntactic rewrite rules that can be applied to optimize the LTL formula before conversion.
Moreover, we integrate the Stuttering Equivalence AFP-Entry by Stefan Merz, adapting the lemma that next-free LTL formula cannot distinguish between
stuttering equivalent runs to our setting.

We use the Isabelle Refinement and Collection framework, as well as the Autoref tool, to obtain a refined version of our algorithm,
from which efficiently executable code can be extracted.
\end{abstract}

\clearpage

\tableofcontents
\clearpage

% sane default for proof documents
\parindent 0pt\parskip 0.5ex


\section{Introduction}

 
We formalize HyperCTL$^*$, a temporal logic for expressing security properties 
introduced in [1,2].  
We first define a shallow embedding of HyperCTL$^*$, within which we prove 
inductive and coinductive rules for the operators.  
Then we show that a HyperCTL$^*$ formula captures Goguen-Meseguer noninterference, a landmark 
information flow property.  
We also define a deep embedding and connect it to the shallow embedding 
by a denotational semantics, for which we prove sanity w.r.t.\ 
dependence on the free variables.  Finally, we show that under some finiteness assumptions 
about the model, 
noninterference is given by a (finitary) syntactic formula.

\par \ \par
For the semantics of HyperCTL$^*$, we mainly follow the earlier paper [1].  
The Kripke structure for representing noninterference is essentially that of [1,Appendix B] -- 
however, instead of using the formula from [1,Appendix B], we further add 
idle transitions to the Kripke structure and use the simpler formula from [1,Section 2.4].  


\par \ \par
[1] Bernd Finkbeiner, Markus N. Rabe and C\'{e}sar S\'{a}nchez.  
A Temporal Logic for Hyperproperties. CoRR, abs$/$1306.6657, 2013.

\par \ \par
[2] Michael R. Clarkson, Bernd Finkbeiner, Masoud Koleini, Kristopher K. Micinski, 
Markus N. Rabe and C\'{e}sar S\'{a}nchez. 
Temporal Logics for Hyperproperties. POST 2014, 265-284.  




% generated text of all theories
\input{session}

% optional bibliography
\bibliographystyle{abbrv}
\bibliography{root}

\end{document}

%%% Local Variables:
%%% mode: latex
%%% TeX-master: t
%%% End:
