\documentclass[11pt,a4paper]{article}
\usepackage{isabelle,isabellesym}

% further packages required for unusual symbols (see also
% isabellesym.sty), use only when needed

%\usepackage{amssymb}
  %for \<leadsto>, \<box>, \<diamond>, \<sqsupset>, \<mho>, \<Join>,
  %\<lhd>, \<lesssim>, \<greatersim>, \<lessapprox>, \<greaterapprox>,
  %\<triangleq>, \<yen>, \<lozenge>

%\usepackage{eurosym}
  %for \<euro>

%\usepackage[only,bigsqcap]{stmaryrd}
  %for \<Sqinter>

%\usepackage{eufrak}
  %for \<AA> ... \<ZZ>, \<aa> ... \<zz> (also included in amssymb)

%\usepackage{textcomp}
  %for \<onequarter>, \<onehalf>, \<threequarters>, \<degree>, \<cent>,
  %\<currency>

% this should be the last package used
\usepackage{pdfsetup}

% urls in roman style, theory text in math-similar italics
\urlstyle{rm}
\isabellestyle{it}

% for uniform font size
%\renewcommand{\isastyle}{\isastyleminor}


\begin{document}

\title{Knot Theory}
\author{T. V. H. Prathamesh}
\maketitle

\begin{abstract}
This work contains a formalization of some topics in knot theory.
The concepts that were formalized include definitions of tangles, links,
framed links and link/tangle equivalence. The formalization is based on a
formulation of links in terms of tangles. We further construct and prove the
invariance of the Bracket polynomial. Bracket polynomial is an invariant of
framed links closely linked to the Jones polynomial. This is perhaps the first
attempt to formalize any aspect of knot theory in an interactive proof assistant.

For further reference, one can refer to the paper "Formalising Knot Theory in Isabelle/HOL" in Interactive Theorem Proving, 6th International Conference, ITP 2015, Nanjing, China, August 24-27, 2015, Proceedings.

\end{abstract}

\tableofcontents

% sane default for proof documents
\parindent 0pt\parskip 0.5ex

% generated text of all theories
\input{session}

% optional bibliography
%\bibliographystyle{abbrv}
%\bibliography{root}

\end{document}

%%% Local Variables:
%%% mode: latex
%%% TeX-master: t
%%% End:
