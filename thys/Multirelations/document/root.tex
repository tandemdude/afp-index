\documentclass[11pt,a4paper]{article}
\usepackage{isabelle,isabellesym}

% this should be the last package used
\usepackage{pdfsetup}

% urls in roman style, theory text in math-similar italics
\urlstyle{rm}
\isabellestyle{it}


\begin{document}

\title{Binary Multirelations}
\author{Hitoshi Furusawa and Georg Struth}

\maketitle

\begin{abstract}
  Binary multirelations associate elements of a set with its subsets;
  hence they are binary relations of type $A\times 2^A$.  Applications
  include alternating automata, models and logics for games, program
  semantics with dual demonic and angelic nondeterministic choices and
  concurrent dynamic logics.  This proof document supports an arXiv
  article that formalises the basic algebra of multirelations and
  proposes axiom systems for them, ranging from weak bi-monoids to
  weak bi-quantales.
\end{abstract}

\tableofcontents

\section{Introduction}

This proof document contains the formal proofs for an article on
\emph{Taming Multirelations}~\cite{FurusawaS15a}.  Individual
cross-references to statements in~\cite{FurusawaS15a} have been added
to this document so that both can be read in parallel.  The first part
of this document contains algebraic axiom systems and equational
proofs.  Some of these proofs are presented in a human-readable style
to indicate the kind of algebraic reasoning involved.  The second part
contains set-theoretic reasoning with concrete multirelations.  Its
main purpose is to justify the algebraic development and to prepare
the soundness proofs of the algebraic axiomatisations with
respect to the concrete multirelational model. Set-theoretic
reasoning with multirelations tends to be very tedious and showing
detailed proofs has not been the aim.

The algebras of multirelations proposed are based on Peleg's
multirelational semantics for concurrent dynamic
logic~\cite{Peleg87}. The most basic axiom systems consider
multirelations under the operations of sequential and concurrent
composition with two corresponding units.  These are enriched by
lattice operations and various fixpoints.  A main source of complexity
is the set-theoretic definition of sequential composition of
multirelations, which is based on higher-order logic. Its use often
requires the Axiom of Choice. In addition, sequential composition is
not associative.

Part of this formalisation is also relevant to a previous approach to
concurrent dynamic algebra by Furusawa and
Struth~\cite{FurusawaS15b}. More material on variants of
multirelations, game algebras and concurrent dynamic algebras will be
added in the future.

The authors are indebted to Alasdair Armstrong and Victor Gomes for
help with some tricky Isabelle proofs.

% sane default for proof documents
\parindent 0pt\parskip 0.5ex

% generated text of all theories
\input{session}

% optional bibliography
\bibliographystyle{plain}
\bibliography{root}

\end{document}

%%% Local Variables:
%%% mode: latex
%%% TeX-master: t
%%% End:
