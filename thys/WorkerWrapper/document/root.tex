\documentclass[11pt,a4paper]{article}
\usepackage{isabelle,isabellesym}
\usepackage{mathpartir}

% Commuting diagrams
\usepackage{pb-diagram}
\usepackage{haskell}

\usepackage{natbib}
\bibpunct();A{},
\let\cite=\citep

\newcommand{\isasymnotsqsubseteq}{\isamath{\not\sqsubseteq}}

\usepackage{amssymb}

\newcommand{\isafun}[1]{{\sf #1}}

% this should be the last package used
\usepackage{pdfsetup}

% urls in roman style, theory text in math-similar italics
\urlstyle{rm}
\isabellestyle{it}


\begin{document}

\title{Mechanising the worker/wrapper transformation}
\author{Peter Gammie}
\maketitle

\tableofcontents

% sane default for proof documents
\parindent 0pt\parskip 0.5ex

\section{Introduction}

This mechanisation of the worker/wrapper theory of
\citet{GillHutton:2009} was carried out in Isabelle/HOLCF
\citep{HOLCF:1999, DBLP:conf/tphol/Huffman09}.  It accompanies
\citet{Gammie:2011}. The reader should note that $oo$ stands for
function composition, $\Lambda \_ . \_$ for continuous function
abstraction, $\_\cdot\_$ for continuous function application,
\textbf{domain} for recursive-datatype definition.

% generated text of all theories
\input{session}

\section{Concluding remarks}

Gill and Hutton provide two examples of fusion: accumulator
introduction in their \S4, and the transformation in their \S7 of an
interpreter for a language with exceptions into one employing
continuations. Both involve strict \<unwrap\>s and are indeed totally
correct.

The example in their \S5 demonstrates the unboxing of numerical
computations using a different worker/wrapper rule and does not
require fusion. In their \S6 a non-strict \<unwrap\> is used to
memoise functions over the natural numbers using the rule considered
here. It should in fact use the same rule as the unboxing example as
the scheme only correctly memoises strict functions. We can see this
by considering a base case missing from their inductive proof, viz
that if \<f :: Nat \to a\> is not strict -- in fact constant, as
\<Nat\> is a flat domain -- then \<f \bot \not= \bot = (map\ f\
[0..])\ !!\ \bot\>, where \<xs\ !!\ n\> is the $n$th element of $xs$.

% optional bibliography
\addcontentsline{toc}{section}{Bibliography}
%\nocite{*}
\bibliographystyle{plainnat}
\bibliography{root}

\end{document}

%%% Local Variables:
%%% mode: latex
%%% TeX-master: t
%%% End:
