\documentclass[11pt,a4paper]{article}
\usepackage{isabelle,isabellesym}
\usepackage{amssymb}

% this should be the last package used
\usepackage{pdfsetup}

% urls in roman style, theory text in math-similar italics
\urlstyle{rm}
\isabellestyle{it}


\begin{document}

\title{A Formalisation of Lehmer's Primality Criterion}
\author{By Simon Wimmer and Lars Noschinski}
\maketitle

\begin{abstract}
  In 1927, Lehmer presented criterions for primality, based on the
  converse of Fermat's litte theorem~\cite{lehmer1927fermat_converse}.
  This work formalizes the second criterion from Lehmer's paper,
  a necessary and sufficient condition for primality.

  As a side product we formalize some properties of Euler's $\varphi$-function,
  the notion of the order of an element of a group, and the cyclicity of the
  multiplicative group of a finite field.
\end{abstract}

\tableofcontents

\section{Introduction}

Section \ref{sec:simp-rules} provides some technical lemmas about polynomials.
Section \ref{sec:euler-phi} to \ref{sec:number-roots} formalize some basic number-theoretic
and algebraic properties: Euler's $\varphi$-function, the order of an element of a group
and an upper bound of the number of roots of a polynomial. Section \ref{sec:mult-group}
combines these results to prove that the multiplicative group of a finite field is cyclic.
Based on that, Section \ref{sec:lehmer} formalizes an extended version of Lehmer's Theorem,
which gives us necessary and sufficient conditions to decide whether a number is prime.

% sane default for proof documents
\parindent 0pt\parskip 0.5ex

% generated text of all theories
\input{session}

\nocite{*}

% optional bibliography
\bibliographystyle{abbrv}
\bibliography{root}

\end{document}

%%% Local Variables:
%%% mode: latex
%%% TeX-master: t
%%% End\dots
