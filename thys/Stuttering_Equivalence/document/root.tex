\documentclass[11pt,a4paper]{article}
\usepackage{isabelle,isabellesym}

% this should be the last package used
\usepackage{pdfsetup}

% urls in roman style, theory text in math-similar italics
\urlstyle{rm}
\isabellestyle{it}


\begin{document}

\title{Stuttering Equivalence and Stuttering Invariance}
\author{
  Stephan Merz\\
  Inria Nancy \& LORIA\\
  Villers-l\`es-Nancy, France
}
\maketitle

\noindent%
Two $\omega$-sequences are stuttering equivalent if they differ only by
finite repetitions of elements. For example, the two sequences
\[
  (abbccca)^{\omega} \qquad\textrm{and}\qquad
  (aaaabc)^{\omega}
\]
are stuttering equivalent, whereas
\[
  (abac)^{\omega} \qquad\textrm{and}\qquad
  (aaaabcc)^{\omega}
\]
are not. Stuttering equivalence is a fundamental concept in the theory
of concurrent and distributed systems. Notably, Lamport~\cite{lamport:what-good}
argues that refinement notions for such systems should be insensitive to
finite stuttering. Peled and Wilke~\cite{peled:ltl-x} showed that all PLTL
(propositional linear-time temporal logic) properties that are insensitive 
to stuttering equivalence can be expressed without the next-time operator. 
Stuttering equivalence is also important for certain verification techniques
such as partial-order reduction for model checking.

We formalize stuttering equivalence in Isabelle/HOL. Our development relies
on the notion of stuttering sampling functions that may skip blocks of 
identical sequence elements. We also encode PLTL and prove the theorem
due to Peled and Wilke~\cite{peled:ltl-x}.


\tableofcontents

% sane default for proof documents
\parindent 0pt\parskip 0.5ex

% generated text of all theories
\input{session}

% optional bibliography
\bibliographystyle{abbrv}
\bibliography{root}

\end{document}

%%% Local Variables:
%%% mode: latex
%%% TeX-master: t
%%% End:
