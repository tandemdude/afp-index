\documentclass[11pt,a4paper]{article}
\usepackage{isabelle,isabellesym}

% this should be the last package used
\usepackage{pdfsetup}

% urls in roman style, theory text in math-similar italics
\urlstyle{rm}
\isabellestyle{it}


\begin{document}

\title{Latin Square}
\author{Alexander Bentkamp}
\maketitle

\begin{abstract}
  A theory about Latin Squares following \cite{aigner}. A Latin Square is a $n \times n$ table filled with
  integers from 1 to n where each number appears exactly once in each row and each column. A Latin Rectangle
  is a partially filled $n \times n$ table with $r$ filled rows and $n-r$ empty rows, such that each number
  appears at most once in each row and each column. The main result of this theory is that any Latin Rectangle
  can be completed to a Latin Square.
\end{abstract}

\tableofcontents

% include generated text of all theories
\input{session}

\bibliographystyle{abbrv}
\bibliography{root}

\end{document}
