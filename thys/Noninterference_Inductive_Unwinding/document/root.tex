\documentclass[11pt,a4paper]{article}
\usepackage{isabelle,isabellesym}
\renewcommand{\isastyletxt}{\isastyletext}

% this should be the last package used
\usepackage{pdfsetup}

% urls in roman style, theory text in math-similar italics
\urlstyle{rm}
\isabellestyle{it}


\begin{document}

\title{The Inductive Unwinding Theorem\\for CSP Noninterference Security}
\author{Pasquale Noce\\Security Certification Specialist at Arjo Systems - Gep S.p.A.\\pasquale dot noce dot lavoro at gmail dot com\\pasquale dot noce at arjowiggins-it dot com}
\maketitle

\begin{abstract}
The necessary and sufficient condition for CSP noninterference security stated
by the Ipurge Unwinding Theorem is expressed in terms of a pair of event lists
varying over the set of process traces. This does not render it suitable for the
subsequent application of rule induction in the case of a process defined
inductively, since rule induction may rather be applied to a single variable
ranging over an inductively defined set.

Starting from the Ipurge Unwinding Theorem, this paper derives a necessary and
sufficient condition for CSP noninterference security that involves a single
event list varying over the set of process traces, and is thus suitable for rule
induction; hence its name, Inductive Unwinding Theorem. Similarly to the Ipurge
Unwinding Theorem, the new theorem only requires to consider individual accepted
and refused events for each process trace, and applies to the general case of a
possibly intransitive noninterference policy. Specific variants of this theorem
are additionally proven for deterministic processes and trace set processes.
\end{abstract}

\tableofcontents

% sane default for proof documents
\parindent 0pt\parskip 0.5ex

% generated text of all theories
\input{session}

% bibliography
\bibliographystyle{abbrv}
\bibliography{root}

\end{document}
