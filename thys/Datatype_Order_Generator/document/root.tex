\documentclass[11pt,a4paper]{article}
\usepackage{isabelle,isabellesym}

% this should be the last package used
\usepackage{pdfsetup}
\usepackage{railsetup}

% urls in roman style, theory text in math-similar italics
\urlstyle{rm}
\isabellestyle{it}

\newcommand\isafor{\textsf{IsaFoR}}
\newcommand\ceta{\textsf{Ce\kern-.18emT\kern-.18emA}}

\begin{document}

\title{Deriving class instances for datatypes.\footnote{Supported by FWF (Austrian Science Fund) project P22767-N13.}}
\author{Ren\'e Thiemann}
\maketitle

\begin{abstract}
  We provide a framework for registering automatic methods 
  to derive class instances 
  of datatypes, 
  as it is possible using Haskell's ``deriving Ord, Show, \ldots'' feature.
  
  We further implemented such automatic methods to derive (linear) orders or
  hash-functions which are required in the 
  Isabelle Collection Framework \cite{rbt} and the Container Framework \cite{containers}. 
  Moreover, for the tactic of
  Huffman and Krauss to show that a datatype is countable, we implemented a 
  wrapper so that this tactic becomes accessible in our framework.
  
  Our formalization was performed as part of the \isafor/\ceta{} project%
  \footnote{\url{http://cl-informatik.uibk.ac.at/software/ceta}} \cite{CeTA}.
  With our new tactic we could completely remove 
  tedious proofs for linear orders of two datatypes.
\end{abstract}

\tableofcontents

\section{Important Information}
The described generators are outdated as they are based on the old datatype package.
Generators for the new datatypes are available in the AFP entry ``Deriving''.

% include generated text of all theories
\input{session}

\section{Acknowledgements}
We thank 
\begin{itemize}
\item Lukas Bulwahn and Brian Huffman for the discussion on a generic derive command and 
  the pointer to
  the tactic for countability.
\item Alexander Krauss for pointing me to the
  recursors of the datatype package.
\item Peter Lammich for the inspiration of developing a hash-function generator.
\item Andreas Lochbihler for the inspiration of developing generators for the container framework.
\item Christian Urban for his cookbook about the ML-level of Isabelle.
\item Stefan Berghofer, Cezary Kaliszyk, and Tobias Nipkow for their explanations
  on several Isabelle related questions.
\end{itemize}

\bibliographystyle{abbrv}
\bibliography{root}

\end{document}
