\documentclass[11pt,a4paper]{article}
\usepackage{isabelle,isabellesym}

% further packages required for unusual symbols (see also
% isabellesym.sty), use only when needed

\usepackage{amssymb}
  %for \<leadsto>, \<box>, \<diamond>, \<sqsupset>, \<mho>, \<Join>,
  %\<lhd>, \<lesssim>, \<greatersim>, \<lessapprox>, \<greaterapprox>,
  %\<triangleq>, \<yen>, \<lozenge>

%\usepackage{eurosym}
  %for \<euro>

\usepackage[only,bigsqcap]{stmaryrd}
 % for \<Sqinter>

%\usepackage{eufrak}
  %for \<AA> ... \<ZZ>, \<aa> ... \<zz> (also included in amssymb)

%\usepackage{textcomp}
  %for \<onequarter>, \<onehalf>, \<threequarters>, \<degree>, \<cent>,
  %\<currency>

% this should be the last package used
\usepackage{pdfsetup}

% urls in roman style, theory text in math-similar italics
\urlstyle{rm}
\isabellestyle{it}

% for uniform font size
%\renewcommand{\isastyle}{\isastyleminor}


\begin{document}

\title{Kleene Algebra}
\author{Alasdair Armstrong, Victor B. F. Gomes, Georg Struth and Tjark Weber}

\maketitle

\begin{abstract}
  Variants of Dioids and Kleene algebras are formalised together with
  their most important models in Isabelle/HOL.  The Kleene algebras
  presented include process algebras based on bisimulation equivalence
  (near Kleene algebras), simulation equivalence (pre-Kleene algebras)
  and language equivalence (Kleene algebras), as well as algebras with
  ambiguous finite or infinite iteration (Conway algebras), possibly
  infinite iteration (demonic refinement algebras), infinite iteration
  (omega algebras) and residuated variants (action algebras).  Models
  implemented include binary relations, (regular) languages, sets of
  paths and traces, power series and matrices.  Finally, min-plus and
  max-plus algebras as well as generalised Hoare logics for Kleene
  algebras and demonic refinement algebras are provided for applications.
\end{abstract}

\tableofcontents

% sane default for proof documents
\parindent 0pt\parskip 0.5ex

\section{Introductory Remarks}

These theory files are intended as a reference formalisation of
variants of Kleene algebras and as a basis for other variants, such as
Kleene algebras with tests~\cite{kat} and modal Kleene
algebras~\cite{kad}, which are useful for program correctness and
verification. To that end we have aimed at making proof accessible to
readers at textbook granularity instead of fully automating them.  In
that sense, these files can be considered a machine-checked
introduction to reasoning in Kleene algebra.

Beyond that, the theories are only sparsely commented. Additional
information on the hierarchy of Kleene algebras and its formalisation
in Isabelle/HOL can be found in a tutorial
paper~\cite{fosterstruthweber11tutorial} or an overview
article~\cite{guttmannstruthweber11tarskikleene}. While these papers
focus on the automation of algebraic reasoning, the present
formalisation presents readable proofs whenever these are interesting
and instructive.

Expansions of the hierarchy to modal Kleene algebras, Kleene algebras
with tests and Hoare logics as well as infinitary and higher-order
Kleene
algebras~\cite{guttmannstruthweber11algmeth,armstrongstruth12hoka},
and an alternative hierarchy of regular algebras and Kleene
algebras~\cite{fosterstruth12regalg}---orthogonal to the present
one---have also been implemented in the Archive of Formal
Proofs~\cite{regalg,kad,kat,rel}.

% generated text of all theories
\input{session}

% optional bibliography
\bibliographystyle{abbrv}
\bibliography{root}

\end{document}

%%% Local Variables:
%%% mode: latex
%%% TeX-master: t
%%% End:
