\documentclass[11pt,a4paper]{article}
\usepackage{isabelle,isabellesym}
\usepackage{amssymb}
\usepackage[english]{babel}

%my own packages
\usepackage{amsmath,amsthm}
\usepackage{mathpartir}
\usepackage{xspace}
\usepackage{multicol}

\usepackage[margin=1.2in]{geometry}

% this should be the last package used
\usepackage{pdfsetup}

% urls in roman style, theory text in math-similar italics
\urlstyle{rm}
\isabellestyle{it}

%Nominal Logic 
\DeclareMathOperator{\finite}{finite}
\DeclareMathOperator{\infinite}{infinite}
\DeclareMathOperator{\supp}{supp}
\DeclareMathOperator{\supports}{~supports~}
\newcommand{\fresh}{~\sharp~}
\newcommand{\A}{\mathbb A}
\newcommand{\N}{\mathbb N}


%Lam_ml notation
\newcommand{\T}{T\,}
\newcommand{\cto}[1]{~\mathsf{to}\ #1 \ \mathsf{in}~}
\newcommand{\+}{+\!\!\!\!+}

\newcommand{\SN}{\ensuremath{\mathit{SN}}}
\newcommand{\red}[1]{\mathit{RED}_{#1}}
\newcommand{\sred}[1]{\mathit{SRED}_{#1}}
\newcommand{\imp}{\Longrightarrow}
\newcommand{\Imp}{\quad \imp \quad}
\newcommand{\Land}{\quad \land \quad}

%HOL-Nominal notation
\newcommand{\pt}{{\textsf{pt}}\ }
\newcommand{\fs}{{\textsf{fs}}\ }


%typesetting for theoremstyle
\newlength{\rulewidth}
\newcommand{\twpage}[1]{\begin{minipage}{\textwidth}#1\end{minipage}}
\newcommand{\rwpage}[1]{\begin{minipage}{\rulewidth}#1\end{minipage}}

%%%%%%%%%%%%%%%%%%%%%%
%Theorems
%%%%%%%%%%%%%%%%%%%%%%

\theoremstyle{plain}
\newtheorem{theorem}{Theorem}[section]
\newtheorem{corollary}[theorem]{Corollary}
\newtheorem{lemma}[theorem]{Lemma}
\newtheorem{proposition}[theorem]{Proposition}

\theoremstyle{definition} 
\newtheorem{definition}[theorem]{Definition}
\newtheorem{property}[theorem]{Property}
\newtheorem{observation}[theorem]{Observation}
\newtheorem{example}[theorem]{Example} 
\newtheorem{counterexample}[theorem]{Counterexample} 

\theoremstyle{remark}
\newtheorem*{notation}{Notation}
\newtheorem*{note}{Note} 
\newtheorem*{proof-attempt}{Proof Attempt}




\title{Strong Normalization of Moggis's Computational Metalanguage}
\author{Christian Doczkal \\ \small{Saarland University}}

\begin{document}
\maketitle

\abstract{
% alpha/binding issues
Handling variable binding is one of the main difficulties in formal proofs.
%
In this context, Moggi's computational metalanguage serves as an interesting case study. It features monadic types and a commuting conversion rule that rearranges the binding structure. 
%
Lindley and Stark have given an elegant proof of strong normalization for this calculus. The key construction in their proof is a notion of relational $\top\top$-lifting, using stacks of elimination contexts to obtain a Girard-Tait style logical relation.

I give a formalization of their proof in Isabelle/HOL-Nominal with a particular emphasis on the treatment of bound variables. 
}

\tableofcontents

% sane default for proof documents
\parindent 0pt\parskip 0.5ex

% generated text of all theories
\input{session}

\section*{Acknowledgments}
I thank Christian Urban, the Nominal Methods group, and the members of the Isabelle mailing list for their helpful answers to my questions.


% optional bibliography
\bibliographystyle{alpha}
\bibliography{root}

\end{document}

%%% Local Variables:
%%% mode: latex
%%% TeX-master: t
%%% End:
