\documentclass[11pt,a4paper]{article}
\usepackage{isabelle,isabellesym}
\usepackage{amssymb}
\usepackage[only,bigsqcap]{stmaryrd}

% this should be the last package used
\usepackage{pdfsetup}

% urls in roman style, theory text in math-similar italics
\urlstyle{rm}
\isabellestyle{it}


\begin{document}

\title{Semantics and Data Refinement of Invariant Based Programs}

\author{Viorel Preoteasa and Ralph-Johan Back}

\maketitle

\begin{abstract}
The invariant based programming is a technique of constructing
correct programs by first identifying the basic situations
(pre- and post-conditions and invariants) that can occur during
the execution of the program, and then defining the transitions
and proving that they preserve the invariants. Data refinement is a 
technique of building correct programs working on concrete datatypes
as refinements of more abstract programs. In the theories presented here 
we formalize the predicate transformer semantics for invariant based 
programs and their data refinement.
\end{abstract}

\tableofcontents

\section{Introduction}
Invariant based programming 
\cite{Back80:invariants,Back83:invariants,aBack08,back:preoteasa:2008} 
is an approach to construct correct programs where we start by 
identifying all basic situations (pre- and post-conditions, and loop 
invariants) that could arise during the execution of the algorithm. 
These situations are determined and described before any code is written. 
After that, we identify the transitions between the situations, which 
together determine the flow of control in the program. The transitions 
are verified at the same time as they are constructed. The correctness 
of the program is thus established as part of the construction process.

These theories present the predicate transformer sematics for invariant based 
programs and their data refinement. The complete treatment of the sematics of 
invariant based programs was presented in 
\cite{back:preoteasa:2008}. There we introduced big and small step semantics, 
predicate transformer semantics, and we proved complete and correct Hoare
rules for invariand based programs. These results were also formalized 
in the PVS theorem prover. In \cite{preoteasa:back:2009} we have studied
data refinement of invariant based programs, and we outlined the steps
for proving the Deutsch-Schorr-Waite marking algorithm using data refinement
of invariant based programs. These theories represent a mechanical 
formalization of the data refinement results from \cite{preoteasa:back:2009}
and some of the results from \cite{back:preoteasa:2008}. In another 
formalization we will show how the theory presented here can be used
in the complete verification of the marking algorithm.


\parindent 0pt\parskip 0.5ex

% generated text of all theories
\input{session}

% optional bibliography
\bibliographystyle{abbrv}
\bibliography{root}

\end{document}

%%% Local Variables:
%%% mode: latex
%%% TeX-master: t
%%% End:
