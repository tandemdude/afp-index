\documentclass[11pt,a4paper]{article}
\usepackage{isabelle,isabellesym}

% further packages required for unusual symbols (see also
% isabellesym.sty), use only when needed

\usepackage{amssymb}
  %for \<leadsto>, \<box>, \<diamond>, \<sqsupset>, \<mho>, \<Join>,
  %\<lhd>, \<lesssim>, \<greatersim>, \<lessapprox>, \<greaterapprox>,
  %\<triangleq>, \<yen>, \<lozenge>

\usepackage[english]{babel}
  %option greek for \<euro>
  %option english (default language) for \<guillemotleft>, \<guillemotright>

%\usepackage[only,bigsqcap]{stmaryrd}
  %for \<Sqinter>

%\usepackage{eufrak}
  %for \<AA> ... \<ZZ>, \<aa> ... \<zz> (also included in amssymb)

%\usepackage{textcomp}
  %for \<onequarter>, \<onehalf>, \<threequarters>, \<degree>, \<cent>,
  %\<currency>

% this should be the last package used
\usepackage{pdfsetup}

% urls in roman style, theory text in math-similar italics
\urlstyle{rm}
\isabellestyle{it}

% for uniform font size
%\renewcommand{\isastyle}{\isastyleminor}


\begin{document}

\title{Regular Algebras}
\author{Simon Foster and Georg Struth}
\maketitle

\begin{abstract}
  Regular algebras axiomatise the equational theory of regular 
  expressions as induced by regular language identity.  We use 
  Isabelle/HOL for a detailed systematic study of regular algebras 
  given by Boffa, Conway, Kozen and Salomaa. We investigate the 
  relationships between these classes, formalise a soundness proof for 
  the smallest class (Salomaa's) and obtain completeness of the 
  largest one (Boffa's) relative to a deep result by Krob. In 
  addition we provide a large collection of regular identities 
  in the general setting of Boffa's axiom. 
\end{abstract}

\tableofcontents

% sane default for proof documents
\parindent 0pt\parskip 0.5ex

\section{Introductory Remarks}

These Isabelle theories complement the article on \emph{On the
  Fine-Structure of Regular Algebra}~\cite{FosterS15}.  For an
introduction to the topic, conceptual explanations and references we
refer to this article. Our regular algebra hierarchy is orthogonal to
the Kleene algebra hierarchy in the Archive of Formal
Proofs~\cite{ArmstrongStruthWeberArchive}; we have not aimed at an
integration for pragmatic reasons.

% generated text of all theories
\input{session}

% optional bibliography
\bibliographystyle{abbrv}
\bibliography{root}

\end{document}

%%% Local Variables:
%%% mode: latex
%%% TeX-master: t
%%% End:
