\documentclass[11pt,a4paper]{article}
\usepackage{isabelle,isabellesym}

\usepackage{url,amssymb,amsthm}

\usepackage{stmaryrd}

% this should be the last package used
\usepackage{pdfsetup}

% urls in roman style, theory text in math-similar italics
\urlstyle{rm}
\isabellestyle{it}

\begin{document}

\title{Secure information flow and program logics --- Isabelle/HOL sources}
\author{Lennart Beringer and Martin Hofmann}

\maketitle

\begin{abstract}
  We present interpretations of type systems for secure
  information flow in Hoare logic, complementing previous encodings in
  relational program logics. We first treat the imperative language
  {\bf IMP}, extended by a simple procedure call mechanism. For this
  language we consider base-line non-interference in the style of
  Volpano et al.~\cite{VolpanoSmithIrvine:JCS1996} and the
  flow-sensitive type system by Hunt and Sands
  \cite{HuntSands:POPL2006}. In both cases, we show how typing
  derivations may be used to automatically generate proofs in the
  program logic that certify the absence of illicit flows. We then add
  instructions for object creation and manipulation, and derive
  appropriate proof rules for base-line non-interference. As a
  consequence of our work, standard verification technology may be
  used for verifying that a concrete program satisfies the
  non-interference property.

  The present proof development represents an update of the
  formalisation underlying our paper~\cite{BeringerHofmann:CSF2007}
  and is intended to resolve any ambiguities that may be present in
  the paper.
\end{abstract}

\tableofcontents

\input{session}

\bibliographystyle{abbrv}
\bibliography{root}

\end{document}
