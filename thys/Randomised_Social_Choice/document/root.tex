\documentclass[11pt,a4paper]{article}
\usepackage{isabelle,isabellesym}

% further packages required for unusual symbols (see also
% isabellesym.sty), use only when needed

%\usepackage{amssymb}
  %for \<leadsto>, \<box>, \<diamond>, \<sqsupset>, \<mho>, \<Join>,
  %\<lhd>, \<lesssim>, \<greatersim>, \<lessapprox>, \<greaterapprox>,
  %\<triangleq>, \<yen>, \<lozenge>

%\usepackage{eurosym}
  %for \<euro>

%\usepackage[only,bigsqcap]{stmaryrd}
  %for \<Sqinter>

%\usepackage{eufrak}
  %for \<AA> ... \<ZZ>, \<aa> ... \<zz> (also included in amssymb)

%\usepackage{textcomp}
  %for \<onequarter>, \<onehalf>, \<threequarters>, \<degree>, \<cent>,
  %\<currency>

% this should be the last package used
\usepackage{pdfsetup}

% urls in roman style, theory text in math-similar italics
\urlstyle{rm}
\isabellestyle{it}

% for uniform font size
%\renewcommand{\isastyle}{\isastyleminor}


\begin{document}

\title{Randomised Social Choice}
\author{Manuel Eberl}
\maketitle

\begin{abstract}
This work contains a formalisation of basic Randomised Social Choice, including Stochastic Dominance
and Social Decision Schemes (SDSs) along with some of their most important properties 
(Anonymity, Neutrality, \textit{SD}-Efficiency, \textit{SD}-Strategy-Proofness) and two particular SDSs -- Random Dictatorship 
and Random Serial Dictatorship (with proofs of the properties that they satisfy). Many important properties of these 
concepts are also proven – such as the two equivalent characterisations of Stochastic Dominance and the fact that 
SD-efficiency of a lottery only depends on the support.

The entry also provides convenient commands to define Preference Profiles, prove their well-formedness, and 
automatically derive restrictions that sufficiently nice SDSs need to satisfy on the defined profiles. (cf. \cite{smt})

Currently, the formalisation focuses on weak preferences and Stochastic Dominance (\textit{SD}), but it should be easy to extend it 
to other domains -- such as strict preferences -- or other lottery extensions -- such as Bilinear Dominance or Pairwise Comparison.
\end{abstract}

\tableofcontents

\parindent 0pt\parskip 0.5ex
\newpage

\input{session}

\bibliographystyle{abbrv}
\bibliography{root}

\end{document}

%%% Local Variables:
%%% mode: latex
%%% TeX-master: t
%%% End:
