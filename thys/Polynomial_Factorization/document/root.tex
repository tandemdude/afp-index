\documentclass[11pt,a4paper]{article}
\usepackage{isabelle,isabellesym}

\usepackage{amssymb}
\usepackage{xspace}

% this should be the last package used
\usepackage{pdfsetup}

% urls in roman style, theory text in math-similar italics
\urlstyle{rm}
\isabellestyle{it}

\newcommand\isafor{\textsf{IsaFoR}}
\newcommand\ceta{\textsf{Ce\kern-.18emT\kern-.18emA}}
\newcommand\rats{\mathbb{Q}}
\newcommand\ints{\mathbb{Z}}
\newcommand\nats{\mathbb{N}}
\newcommand\reals{\mathbb{R}}
\newcommand\mod{\mathit{mod}}
\newcommand\complex{\mathbb{C}}

\newcommand\rai{real algebraic number\xspace}
\newcommand\rais{real algebraic numbers\xspace}

\begin{document}

\title{Polynomial Factorization\footnote{Supported by FWF (Austrian Science Fund) project Y757.}}
\author{Ren\'e Thiemann and Akihisa Yamada}
\maketitle

\begin{abstract}
Based on existing libraries for polynomial interpolation and matrices,
we formalized several factorization algorithms for polynomials, including
Kronecker's algorithm for integer polynomials,
Yun's square-free factorization algorithm for field polynomials, and
Berlekamp's algorithm for polynomials over finite fields.
By combining the last one with Hensel's lifting,
we derive an efficient factorization algorithm for the integer polynomials,
which is then lifted for rational polynomials by mechanizing Gauss' lemma.
Finally, we assembled a combined factorization algorithm for rational polynomials,
which combines all the mentioned algorithms and additionally uses the explicit formula for roots 
of quadratic polynomials and a rational root test.

As side products, we developed division algorithms for polynomials over integral domains,
as well as primality-testing and prime-factorization algorithms for integers.
\end{abstract}

\tableofcontents

\section{Introduction}

The details of the factorization algorithms have mostly been extracted 
from Knuth's Art of Computer Programming
\cite{Knuth}. Also Wikipedia provided valuable help.

\medskip
As a first fast
preprocessing for factorization we integrated Yun's factorization algorithm which identifies duplicate
factors \cite{Yun}. In contrast to the existing formalized result that the GCD of $p$ and $p'$ has no
duplicate factors (and the same roots as $p$), Yun's algorithm decomposes a polynomial $p$ into
$p_1^1 \cdot \ldots \cdot p_n^n$ such that no $p_i$ has a duplicate factor and there is no common
factor of $p_i$ and $p_j$ for $i \neq j$. As a comparison, the GCD of $p$ and $p'$ is exactly
$p_1 \cdot \ldots \cdot p_n$, but without decomposing this product into the list of $p_i$'s.

Factorization over $\rats$ is reduced to factorization over $\ints$ 
with the help of Gauss' Lemma.

Kronecker's algorithm for factorization over $\ints$ requires both
polynomial interpolation over $\ints$ and prime factorization over $\nats$. Whereas the former
is available as a separate AFP-entry, for prime factorization we mechanized a simple algorithm depicted
in \cite{Knuth}:
For a given number $n$,
the algorithm iteratively checks divisibility by numbers until $\sqrt n$,
with some optimizations:
it uses a precomputed set of small primes (all primes up to 1000), 
and if $n\ \mod\ 30 = 11$, the next test candidates in the range $[n,n+30)$ 
are only the 8 numbers $n,n+2,n+6,n+8,n+12,n+18,n+20,n+26$.

However, in theory and praxis it turned out that Kronecker's algorithm is too inefficient. 
Therefore, we implemented the much faster Berlekamp factorization algorithm which works over finite fields \cite{Berlekamp}, and
then lifts the factorization to $\ints$ via the Hensel lifting \cite{Hensel}. However, the Berlekamp-Hensel
factorization is only available as oracle; it is checked at runtime that the resulting
factorization is valid (the input polynomial is the same as the product of factors), 
but no results on the optimality are provided (all factors are irreducible). 
Therefore, if irreducibility of the resulting factors has to be formally ensured, 
Kronecker's algorithm is applied to validate irreducibility of the generated factors.

Although only being an oracle, we tried to include several certified parts in the implementation
of the Berlekamp-Hensel algorithm.
\begin{itemize}
\item In Berlekamp's algorithm, a basis for a set of vectors over some finite field has to be determined.
  To this end, we generalized the certified Gauss-Jordan algorithm from \cite{JNF-AFP} so that 
  it takes the field operations like addition, multiplication, and inverse as parameter. 
  As a consequence, we can just invoke the certified algorithm 
  parametrized for finite fields. Intuitively, then nothing can go wrong; however, this is not
  formally stated, since the soundness of Gauss-Jordan was proven for type-based operations, whereas
  the soundness of the finite-field operations can only be proven set-based, since the cardinality of the
  carrier has to be dynamically determined at runtime.
\item For operations on polynomials, we took a similar but not identical approach. Instead of 
  generalizing the algorithms on polynomials to take the basic field operations as parameters, we just
  copied the generated code for polynomials and manually replaced all basic operations by parameters.
  So, the algorithms indeed should be sound, although nothing has been proven about it.
  
  As a side effect, we inspected all the generated code equations for polynomials and optimized some of them,
  cf.\ \texttt{Improved-Code-Equations} in \texttt{../Polynomial-Interpolation}.
\end{itemize}

The combined factorization algorithm first factorizes the polynomial
via Yun and Berlekamp-Hensel. For each constructed factor, the algorithm 
certifies irreducibility as follows: For polynomials of degree 2, the 
closed form for the roots of quadratic polynomials is applied. For polynomials of degree 3, 
the rational root test determines whether the polynomial is irreducible or not, and finally
for degree 4 and higher, Kronecker's factorization algorithm is applied.

As future work it would be beneficial to perform a full formalization of 
the Berlekamp-Hensel factorization algorithm as it is by far more efficient than Kronecker's algorithm.
\input{session}



\bibliographystyle{abbrv}
\bibliography{root}

\end{document}
