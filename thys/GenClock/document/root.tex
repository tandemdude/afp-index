\documentclass[11pt,a4paper]{article}
\usepackage{fullpage}
\usepackage{isabelle,isabellesym}

% this should be the last package used
\usepackage{pdfsetup}

% urls in roman style, theory text in math-similar italics
\urlstyle{rm}
\isabellestyle{it}


\begin{document}

\title{Formalization of a Generalized Protocol for 
Clock Synchronization in Isabelle/HOL}
\author{Alwen Tiu \\
LORIA - \url{http://qsl.loria.fr}}
\maketitle


\begin{abstract}
We formalize the generalized Byzantine fault-tolerant clock synchronization 
protocol of Schneider. 
This protocol abstracts from particular algorithms or implementations
for clock synchronization. This abstraction includes several
assumptions on the behaviors of physical clocks and on general
properties of concrete algorithms/implementations. Based on these assumptions
the correctness of the protocol is proved by Schneider. 
His proof was later verified by Shankar using the theorem prover EHDM
(precursor to PVS). Our formalization in Isabelle/HOL is based on
Shankar's formalization. 
\end{abstract}

\tableofcontents

\parindent 0pt\parskip 0.5ex

\section{Introduction}

In certain distributed systems, e.g., real-time process-control
systems, the existence of a reliable global time source is critical in
ensuring the correct functioning of the systems.
This reliable global time source can be implemented using
several physical clocks distributed on different nodes in the
distributed system. Since physical clocks are by nature constantly
drifting away from the ``real time'' and different clocks can have
different drift rates, in such a scheme, it is important that these
clocks are regularly adjusted so that they are closely synchronized
within a certain application-specific safe bound. The design and
verification of clock synchronization protocols are often complicated
by the additional requirement that the protocols should work correctly
under certain types of errors, e.g., failure of some clocks, error in
communication network or corrupted messages, etc.

There has been a number of fault-tolerant clock synchronization
algorithms studied in the literature, e.g., the {\em Interactive
  Convergence Algorithm} ({ICA}) by Lamport and
Melliar-Smith~\cite{Lamport}, the Lundelius-Lynch algorithm
\cite{Lundelius}, etc., each with its own degree of fault tolerance.
One important property that must be satisfied by a clock
synchronization algorithm is the agreement property, i.e., at any time
$t$, the difference of the clock readings of any two non-faulty processes must be
bounded by a constant (which is fixed according to the domain of
applications).  At the core of these algorithms is the convergence
function that calculates the adjustment to a clock of a process, based
on the clock readings of all other processes.
Schneider~\cite{Schneider87} gives an abstract
characterization of a wide range of clock synchronization algorithms
(based on the convergence functions used) and proves the agreement
property in this abstract framework.  Schneider's proof was later
verified by Shankar \cite{Shankar92} in the theorem prover EHDM
(precursor to PVS), where eleven axioms about clocks are explicitly stated. 

We formalize Schneider's proof in Isabelle/HOL, making use of Shankar's formulation of the
clock axioms. The particular formulation of axioms on clock
conditions and the statements of the main theorems here are
essentially those of Shankar's \cite{Shankar92}, with some minor
changes in syntax. For the full description of the protocol, the
general structure of the proof and the meaning of the constants and
function symbols used in this formalization, we refer readers to \cite{Shankar92}.

\paragraph{Acknowledgment} 
I would like to thank Stephan Merz and Pascal Fontaine for useful tips on using Isabelle
and particularly the Isar proof language.
 
\section{Isar proof scripts}

% include generated text of all theories
\input{session}

\bibliographystyle{abbrv}
\bibliography{root}

\end{document}
