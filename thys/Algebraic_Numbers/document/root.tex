\documentclass[11pt,a4paper]{article}
\usepackage{isabelle,isabellesym}

\usepackage{amssymb}
\usepackage{xspace}

% this should be the last package used
\usepackage{pdfsetup}

% urls in roman style, theory text in math-similar italics
\urlstyle{rm}
\isabellestyle{it}

\newcommand\isafor{\textsf{IsaFoR}}
\newcommand\ceta{\textsf{Ce\kern-.18emT\kern-.18emA}}
\newcommand\rats{\mathbb{Q}}
\newcommand\ints{\mathbb{Z}}
\newcommand\reals{\mathbb{R}}
\newcommand\complex{\mathbb{C}}

\newcommand\rai{real algebraic number\xspace}
\newcommand\rais{real algebraic numbers\xspace}

\begin{document}

\title{Algebraic Numbers in Isabelle/HOL\footnote{Supported by FWF (Austrian Science Fund) project Y757.}}
\author{Ren\'e Thiemann and Akihisa Yamada}
\maketitle

\begin{abstract}
Based on existing libraries for matrices, factorization of rational polynomials, 
and Sturm's theorem,
we formalized algebraic numbers in Isabelle/HOL. Our development serves
as an implementation for real and complex numbers, and it admits to compute roots
and completely factorize real and complex polynomials, provided that all
coefficients are rational numbers. Moreover, we provide two implementations to display
algebraic numbers, an injective and expensive one, or a faster but approximative version.

To this end, we mechanized several results on resultants,
which also required us to prove that polynomials over a unique factorization domain form 
again a unique factorization domain. 
\end{abstract}

\tableofcontents

\section{Introduction}

Isabelle's previous implementation of irrational numbers was limited:
it only admitted numbers expressed in the form ``$a+b\sqrt{c}$'' for $a,b,c \in \rats$,
and even computations like $\sqrt2 \cdot \sqrt3$ led to a runtime error \cite{Real-AFP}.

In this work, we provide full support for the \emph{real algebraic numbers},
i.e., the real numbers that are expressed as roots of non-zero rational polynomials,
and we also partially support complex algebraic numbers.

Most of the results on algebraic numbers have been taken from a textbook 
by Bhubaneswar Mishra \cite{AlgNumbers}.
Also Wikipedia provided valuable help.

\medskip
Concerning the real algebraic numbers, we first had to prove that they form a field.
To show that the addition and multiplication of \rais are also \rais,
we formalize the theory of \emph{resultants}, which are the determinants of 
specific matrices, where the size of these matrices depend on the degree of the 
polynomials. To this end, we utilized the matrix library provided in the Jordan-Normal-Form
AFP-entry \cite{JNF-AFP} where the matrix dimension can arbitrarily be chosen at runtime.

Given \rais $x$ and $y$ expressed as the roots of polynomials,
we compute a polynomial that has $x+y$ or $x \cdot y$ as its root via resultants.
In order to guarantee that the resulting polynomial is non-zero, we needed the result
that multivariate polynomials over fields form a unique factorization domain (UFD).
To this end, we proved that polynomials over some UFD are again a UFD, relying
upon results in HOL-algebra.

When performing actual computations with algebraic numbers, it is important to reduce
the degree of the representing polynomials. To this end, we use existing
formalized factorization algorithms. At the moment, there is a fast oracle for factorization
available (which does not guarantee optimality of the factorization), and a less efficient
but provably complete factorization algorithm. To this end, currently all but one algorithms are
based on the efficient oracle for factorization. Only, the default show-function for real algebraic numbers requires the unique minimal polynomial representing the algebraic number --
but an alternative which displays only an approximative value is also available.

In order to support tests on whether a given algebraic number is a rational number, 
it would of course also suffice to first compute the minimal polynomial, and then test 
whether the degree is one. Instead, here we just perform the rational root test.

The formalization of Sturm's method \cite{Sturm-AFP} was crucial to separate the different
roots of a fixed polynomial. 
We could nearly use it as it is, and just copied some function definition so
that Sturm's method now is available to separate the real roots of rational polynomial, where
all computations are now performed over $\rats$.

With all the mentioned ingredients we implemented all arithmetic operations on real algebraic
numbers, i.e., addition, subtraction, multiplication, division, comparison, $n$-th root, floor-
and ceiling, and testing on membership in $\rats$. Moreover, we provide a method
to create real algebraic numbers from a given rational polynomial, a method which computes
precisely the set of real roots of a rational polynomial.

\medskip

The absence of an equivalent to Sturm's method for the complex numbers in Isabelle/HOL prevented
us from having native support for complex algebraic numbers. Instead, we represent complex
algebraic numbers as their real and imaginary part: note that a complex number is algebraic if
and only if both the real and the imaginary part are real algebraic numbers.
This equivalence also admitted us to design an algorithm which computes all complex roots
of a rational polynomial. It first constructs a set of polynomials which represent all
real and imaginary parts of all complex roots, yielding a superset of all roots,
and afterwards the set just is just filtered.

By the fundamental theorem of algebra, we then also have a factorization algorithm for
polynomials over $\complex$ with rational coefficients.

Finally, for factorizing a rational polynomial over $\reals$, we first factorize it over $\complex$,
and then combine each pair of complex conjugate roots.

\medskip

As future work it would be beneficial to have a soundness proof for the rational factorization oracle, since then some of the operations on real algebraic numbers can be executed more 
efficiently. For instance, we know that internally always a unique representation of 
real algebraic numbers is used, which would admit fast equality checks. But currently we cannot use this knowledge and have to decide equality via Sturm's method.

Moreover, we did not include the result that the set of complex algebraic numbers is algebraically
closed, i.e., we are limited to determine the complex roots of a polynomial over $\rats$, and
cannot determine the real or complex roots of an polynomial having arbitrary algebraic coefficients.

Finally, an analog to Sturm's method for the complex numbers would be welcome, in order to
have a smaller representation: for instance, currently the complex roots of $1 + x + x^3$ are
computed as ``root \#1 of $1 + x + x^3$'', 
``(root \#1 of $-\frac18 + \frac14x + x^3$)+(root \#1 of $-\frac{31}{64} + \frac{9}{16}x^2 - \frac32x^4 + x^6$)i'', and 
``(root \#1 of $-\frac18 + \frac14x + x^3$)+(root \#2 of $-\frac{31}{64} + \frac{9}{16}x^2 - \frac32x^4 + x^6$)i''.

\section{Auxiliary Algorithms}


% include generated text of all theories
\input{session}



\bibliographystyle{abbrv}
\bibliography{root}

\end{document}
