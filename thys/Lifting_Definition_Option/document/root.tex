%    Title:       Lifting Definition Option
%    Author:      René Thiemann       <rene.thiemann@uibk.ac.at>
%    Maintainer:  René Thiemann
%    License:     LGPL
%
%Copyright 2014 René Thiemann
%
%This file is part of IsaFoR/CeTA.
%
%IsaFoR/CeTA is free software: you can redistribute it and/or modify it under the
%terms of the GNU Lesser General Public License as published by the Free Software
%Foundation, either version 3 of the License, or (at your option) any later
%version.
%
%IsaFoR/CeTA is distributed in the hope that it will be useful, but WITHOUT ANY
%WARRANTY; without even the implied warranty of MERCHANTABILITY or FITNESS FOR A
%PARTICULAR PURPOSE.  See the GNU Lesser General Public License for more details.
%
%You should have received a copy of the GNU Lesser General Public License along
%with IsaFoR/CeTA. If not, see <http://www.gnu.org/licenses/>.
%
\documentclass[11pt,a4paper]{article}
\usepackage{isabelle,isabellesym}

\usepackage{amsmath}
\usepackage{xspace}

% this should be the last package used
\usepackage{pdfsetup}

% urls in roman style, theory text in math-similar italics
\urlstyle{rm}
\isabellestyle{it}

\newcommand\isakwd[1]{\textsf{\isa{#1}}}
\newcommand\isasimpmp{\isa{simplify-emp-main}}
\newcommand\parfun{\isakwd{partial-function}}
\newcommand\vect[1]{\overrightarrow{#1}}
\newcommand\fs{\isa{fs}}
\newcommand\xs{\isa{xs}}
\newcommand\xst{\isa{xs}_t}
\newcommand\inT{\isa{in}}
\newcommand\cprod[1]{({#1})}
\newcommand\outT{\isa{out}}
\newcommand\monad{\isa{monad}}
\newcommand\tto\Rightarrow
\newcommand\ar{\isa{ar}}
\newcommand\inj{\isa{inj}}
\newcommand\proj{\isa{proj}}
\newcommand\mapM{\isa{map-monad}}
\newcommand\curry{\isa{curry}}
\newcommand\case{\isakwd{case}}
\newcommand\of{\isakwd{of}}
\newcommand\ldo{\isacommand{lift{\isacharunderscore}definition{\isacharunderscore}option}}
\newcommand\ld{\isacommand{lift{\isacharunderscore}definition}}
\newcommand\ys{y\isactrlsub {\isadigit{1}}\,\ldots\,y\isactrlsub n}

\newcommand\isafor{\textsf{Isa\kern-0.2exF\kern-0.2exo\kern-0.2exR}\xspace}
\newcommand\ceta{\textsf{C\kern-0.2exe\kern-0.5exT\kern-0.5exA}\xspace}


\begin{document}

\title{Lifting Definition Option\thanks{This research is supported by FWF (Austrian Science Fund) project Y 757.}}
\author{Ren\'e Thiemann}
\maketitle

\begin{abstract}
  We implemented a command, \ldo, which can be used to easily generate
  elements of a restricted type  
  \isa{{\isacharbraceleft}x\ {\isacharcolon}{\isacharcolon}\ {\isacharprime}a{\isachardot}\ P\ x{\isacharbraceright}},
  provided the definition is of the form
  \isa{\isasymlambda\ \ys{\isachardot}\ if\ check\ \ys\ then\ Some\ {\isacharparenleft}generate\ \ys\ {\isacharcolon}{\isacharcolon}\ {\isacharprime}a{\isacharparenright}\ else\ None} and
\isa{check\ \ys\ {\isasymLongrightarrow}\ P\ {\isacharparenleft}generate\ \ys{\isacharparenright}}
 can be proven.

  In principle,
  such a definition is also directly possible using one invocation of \ld. However, then
  this definition will not be suitable for code-generation. To this end, we automated 
  a more complex construction of Joachim Breitner which is amenable for code-generation,
  and where the test \isa{check\ \ys} will only be performed once.
  In the automation, one auxiliary type is created, 
  and Isabelle's lifting- and transfer-package is invoked several
  times. 
\end{abstract} 

\textbf{This entry is outdated as in the meantime the lifting- and transfer-package 
has the desired functionality in an even more general way. Therefore, only the examples
are kept.}
 
\tableofcontents

\medskip 

\input{session}

\subsection*{Acknowledgements}
We thank Andreas Lochbihler for pointing us to Joachim's solution,
and we thank Makarius Wenzel for explaining us, how we can go back
from states to local theories within Isabelle/ML. 
\bibliographystyle{abbrv}
\bibliography{root}
\end{document}
